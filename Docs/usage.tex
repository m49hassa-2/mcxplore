\section{Usage}

%run [-m model |[-h]
%	echo "-m, --model		specify which model to use: \"CMDmdl\" for Model Checking Command model, \"REqmdl\" for Model Checking request model, or \"noMChk\" for the no model checking usage. noMChk is the default"
%	echo "-o, --out		  specify the output file name of the generated test. results/[model]/Test.trc is the default"
%	echo "-h, --help		prints out this usage message"          
The usage of \textit{MCXplore} is as simple as invoking the \textit{\textbf{MCXplore.sh}} script as follows:
\begin{lstlisting}[language=bash]
./ mcxplore [[-m model] [-o output_file] [-t DRAM_timing_file] [-s LTL_specs_file] |[-h]]

\end{lstlisting}
Table~\ref{tb:usage} illustrates the possible options and their description.
\begin{table}[h] 
\scriptsize
\centering
\caption{Usage arguments.\label{tb:usage}}
\begin{tabular}{|l|l|l|p{5cm}|p{5cm}|}
  \hline
   Short flag &  Long flag & \multicolumn{2}{l|}{Options} & Description\\
   \hline
  \multirow{3}{*}{-m} & \multirow{3}{*}{-{}-model} & CMDmdl & Model Checking Command model &   \multirow{3}{*}{Specify which model to use}\\ 
  
  & & REQmdl & Model Checking request model &\\
  & &noMChk& no model checking usage. noMChk is the default&\\
      
\hline


-o & -{}-out& \multicolumn{2}{l|}{Any valid file name}&specify the output file name of the generated test. results/[model]/Test.trc is the default\\

\hline


-t &  -{}-DDrTiming & \multicolumn{2}{l|}{Any valid file name} & specify the input timing file with DDR constraints. DDrTimings/DDR3\_1600.tim is the default\\

\hline

-s &  -{}-Spec &  \multicolumn{2}{l|}{Any valid file name} &	specify the input LTL specification file that models the test plan. LTLspec/CMDmdl.spec is the default\\

\hline

-h & -{}-help & \multicolumn{2}{l|}{No options}&prints out this usage message\\

\hline
\end{tabular}
\label{tb:vars}
\end{table}
