\externaldocument{configuration_tbl}
\externaldocument{flow}
\externaldocument{TL}
\section{Introduction}
\textit{MCXplore} is an automation tool that generates tests to validate the memory system and evaluate its performance. 
The novel observation behind \textit{MCXplore} is the following. 
Regardless of various policies proposed for the memory system, the test pattern is still the same. 
This test pattern consists of a sequence of memory requests. 
Each request has three major components: the address, the transaction type (whether read or write), and the transaction size (the number of bytes to transfer). 
Based on this observation, instead of building a validation framework per memory policy, \textit{MCXplore} relies in the input test pattern to validate any memory policy. 
\textit{MCXplore} has two modes of operation: non model checking mode and model-checking mode.
\subsection{Non model checking mode}
In the first one, the user configures a set of parameters to generate the test that has the desired properties. 
These parameters represent the pattern required for the address and type of transactions in the test. 
This mode is suitable for validating common memory controller aspects such as:
\begin{enumerate}
\item Common page policies such as close-, open- and adaptive-page policies. 

\item Common address mapping schemes such as traditional address mappings with any segment order. For example, CH:RW:RNK:BNK:CL, CH:RW:BNK,RNK,CL, ..., etc. 
It is also suitable for non traditional mappings such as XOR address mapping, where bank bits are XORed with low-significant row bits. 

\end{enumerate}  

This mode has a set of predefined-- yet expandable-- patterns for each of the address segments as well as the transaction type. 
We tabulate currently implemented patterns in Table~\ref{tb:config_noMChk}. 
Although this model is easy to use, it does not necessarily cover the whole state space of the test pattern. 
%Consequently, we propose the second mode. 

To use this mode, the user does not need to specify any model to the tool as this mode is the default for \textit{MCXplore}. 
Alternatively, the user can specify this mode by setting the -m flag to \textbf{noMChk}. 
As the flow diagram in Figure~\ref{fig:flow} illustrates, the main script for this mode is \textit{\textbf{MCXplore\_noMChk.pl}}.



\subsection{Model Checking mode}
The second mode leverages model checking capabilities to cover the state space of the memory input test. 
Model checking automates the state-space exploration of the test generation,
and provides a formal methodology to define test properties.
We create two abstract models to express the stimulus test of the MC: a request interrelation model, \textbf{REQmdl}, and a command interaction model, \textbf{CMDmdl}, and we encode them as FSMs in the NuSMV model checker. The two NuSMV models are in \textbf{\textit{CMDmdl.smv}} and \textbf{\textit{REQmdl.smv}} under the \textbf{\textit{\textbackslash Models}} directory, respectively. 
the user is able to encode test properties as specifications expressed in temporal logic formulas. 
Formulas are negated such that they are true if required test properties do not exist. 
For examples of these formulas, see Section~\ref{sec:TL}.
As the flow diagram in Figure~\ref{fig:flow} illustrates, the main two script for these  two models are \textit{\textbf{MCXplore\_REQmdl.sh}} and \textit{\textbf{MCXplore\_CMDmdl.sh}}, respectively.
